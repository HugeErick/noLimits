% arara: pdflatex: { shell: yes }
\documentclass[twoside]{article}
\usepackage[utf8]{inputenc}
\usepackage[english]{babel}
\usepackage{amsmath, amssymb, amsthm}
\usepackage{hyperref}
\usepackage{ragged2e}
\usepackage{graphicx}
\usepackage{float}
\usepackage{fancyhdr}
\usepackage{geometry}
\usepackage{multicol}
\usepackage{url}
\usepackage{listings} % For better code formatting
\usepackage{xcolor} % For syntax highlighting

% Suppress underfull and overfull warnings
\tolerance=1000
\emergencystretch=10pt

\setlength{\headheight}{15.2pt}
\geometry{paperwidth=8.5in, paperheight=11.0in, top=1.0in, bottom=1.0in, left=1.0in, right=1.0in}

\pagestyle{fancyplain}
\fancyhead[LO]{Activity \#2.1}
\fancyhead[CO]{}
\fancyhead[RO]{P25-LIS-3012}
\fancyfoot[LO]{\thepage}
\fancyfoot[CO]{Advanced Databases, UDLAP}
\fancyfoot[RO]{}

% Define a style for XQuery code
\lstdefinestyle{xquery}{
    basicstyle=\ttfamily\small,
    breaklines=true, % Allow line breaks
    breakatwhitespace=true, % Break lines at spaces
    frame=single, % Add a frame around the code
    captionpos=b, % Caption position
    keywordstyle=\color{blue}, % Keywords in blue
    commentstyle=\color{green}, % Comments in green
    stringstyle=\color{red}, % Strings in red
    numbers=left, % Line numbers on the left
    numberstyle=\tiny\color{gray}, % Line numbers style
    tabsize=2, % Tab size
    showstringspaces=false % Don't show spaces in strings
}

\begin{document}

\fancypagestyle{plain}{
    \renewcommand{\headrulewidth}{1pt}
    \renewcommand{\footrulewidth}{1pt}
}

\title{XQuery Language}
\author{\small{Erick Gonzalez Parada ID: 178145}\\
\small{Emiliano Ruiz Plancarte ID: 177478} \\
\small{Andre Francois Duhamel Gutierrez ID: 177315} \\
\small{Antonio Gutiérrez Blanco ID: 177442}}
\date{\today}
\maketitle

\begin{abstract}
    \raggedright
    This document explores the fundamentals of the XQuery language, showcasing query examples, methodologies, and conclusions drawn from practical applications.
\end{abstract}

\begin{justify}
    \textbf{\textit{Keywords:}} XML, DTD, XPath, XQuery, BaseX.
\end{justify}

\section{Theoretical Framework}
XQuery is a powerful and flexible query language designed for querying and transforming XML data. It is often referred to as "SQL for XML" due to its ability to extract and manipulate data stored in XML documents. XQuery is built on XPath expressions and provides additional features such as FLWOR (For, Let, Where, Order by, Return) expressions, which allow for complex data retrieval and transformation tasks \cite{w3}.

One of the key strengths of XQuery is its ability to handle hierarchical and nested data structures, which are common in XML documents. This makes it particularly useful for applications such as web services, data integration, and content management systems. For example, XQuery can be used to extract specific elements from an XML document, transform the data into a different format, or generate reports based on the data \cite{microsoft}.

Another important feature of XQuery is its support for strong typing and schema validation. This ensures that the data being queried adheres to a predefined structure, reducing the risk of errors and improving the reliability of the queries. Additionally, XQuery supports modularity, allowing developers to create reusable modules and functions that can be shared across different projects \cite{w3schools}.

\subsection*{Goals}
The goal of this lab is to have a first encounter with the XQuery language to examine XML
documents.

\subsection*{Materials}
\begin{itemize}
\item \textit{BaseX}
\end{itemize}

\section{Methodology}
The methodology involves writing and executing XQuery scripts to perform the operations on the XML documents. Each query is designed to demonstrate a specific feature or use case of XQuery, such as filtering, sorting, grouping, and transforming data. The results of each query are analyzed to understand the underlying principles and techniques.

\section{Query Results}
Below are the XQuery scripts and their corresponding results:

\begin{enumerate}
%1
    \item List books published by "Addison-Wesley" after 1991:
    \begin{lstlisting}[style=xquery]
    let $books := doc("bookshop.xml")//book
    where $books/publisher = "Addison-Wesley" and $books/@year > 1991
    return <book><year>{$books/@year}</year><title>{$books/title/text()}</title></book>
    \end{lstlisting}
%2
    \item Create a flattened list of "result" items with title and author:
    \begin{lstlisting}[style=xquery]
    for $book in doc("bookshop.xml")//book,
        $author in $book/author
    return <result><title>{$book/title/text()}</title><author>{$author/name/text()} {$author/lastname/text()}</author></result>
    \end{lstlisting}
%3
    \item List titles with grouped authors:
    \begin{lstlisting}[style=xquery]
    for $book in doc("bookshop.xml")//book
    return <result>
        <title>{$book/title/text()}</title>
        <authors>{
            for $author in $book/author
            return <author>{$author/name/text()} {$author/lastname/text()}</author>
        }</authors>
    </result>
    \end{lstlisting}
%4
    \item List authors with their books:
    \begin{lstlisting}[style=xquery]
    for $author in distinct-values(doc("bookshop.xml")//author)
    let $books := doc("bookshop.xml")//book[author/name = $author/name and author/lastname = $author/lastname]
    return <result>
        <author>{$author/name/text()} {$author/lastname/text()}</author>
        <titles>{
            for $book in $books
            return <title>{$book/title/text()}</title>
        }</titles>
    </result>
    \end{lstlisting}
%5
    \item Books with number of authors:
    \begin{lstlisting}[style=xquery]
    for $book in doc("bookshop.xml")//book
    return <book>
        <title>{$book/title/text()}</title>
        {if ($book/author) then <number-of-authors>{count($book/author)}</number-of-authors> else ()}
    </book>
    \end{lstlisting}
%6
    \item Find minimum price for each book in prices.xml:
    \begin{lstlisting}[style=xquery]
    for $title in distinct-values(doc("prices.xml")//book/title)
    let $minPrice := min(doc("prices.xml")//book[title = $title]/price)
    return <minimum-price title="{$title}">{$minPrice}</minimum-price>
    \end{lstlisting}
%7
    \item Alphabetically list Addison-Wesley books after 1991:
    \begin{lstlisting}[style=xquery]
    for $book in doc("bookshop.xml")//book
    where $book/publisher = "Addison-Wesley" and $book/@year > 1991
    order by $book/title
    return <book>
        <title>{$book/title/text()}</title>
        <year>{$book/@year}</year>
    </book>
    \end{lstlisting}
%8
    \item Return book element for authored books, reference for published ones:
    \begin{lstlisting}[style=xquery]
    for $book in doc("bookshop.xml")//book
    return
        if ($book/author)
        then <book>
            <title>{$book/title/text()}</title>
            <authors>{
                for $author in $book/author
                return <author>{$author/name/text()} {$author/lastname/text()}</author>
            }</authors>
        </book>
        else if ($book/editor)
        then <reference>
            <title>{$book/title/text()}</title>
            <affiliation>{$book/editor/affiliation/text()}</affiliation>
        </reference>
        else ()
    \end{lstlisting}
%9
    \item List book titles with their prices at each bookshop:
    \begin{lstlisting}[style=xquery]
    for $book in distinct-values(doc("prices.xml")//book/title)
    return <book>
        <title>{$book/text()}</title>
        {
            for $store in doc("prices.xml")//book[title = $book]
            return <price source="{$store/source/text()}">{$store/price/text()}</price>
        }
    </book>
    \end{lstlisting}
\end{enumerate}
\section{Conclusions}
Through this exercise, we gained a deeper understanding of the XQuery language and its capabilities in querying and transforming XML data. The practical examples demonstrated the flexibility and power of XQuery in handling complex data retrieval and manipulation tasks. Future work could explore advanced features such as XQuery functions, modules, and integration with other technologies, will ORACLE save us from a painfull dev experiencie with XML?

\begin{thebibliography}{9}
    \bibitem{w3}
    XQuery 3.1: An XML query language. (n.d.). Www.w3.org. Retrieved March 15, 2025, from \url{https://www.w3.org/TR/xquery-31/}
    \bibitem{microsoft}
    XQuery language reference (SQL Server). (n.d.). Microsoft.com. Retrieved March 15, 2025, from \url{https://learn.microsoft.com/en-us/sql/xquery/xquery-language-reference-sql-server?view=sql-server-ver16}
    \bibitem{w3schools}
    XQuery tutorial. (n.d.). W3schools.com. Retrieved March 15, 2025, from \url{https://www.w3schools.com/xml/xquery_intro.asp}
\end{thebibliography}

\end{document}
