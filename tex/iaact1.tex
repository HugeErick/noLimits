% arara: pdflatex: { shell: yes }
\documentclass[twoside]{article}
\usepackage[utf8]{inputenc}
\usepackage[english]{babel}
\usepackage{amsmath}
\usepackage{graphicx}
\usepackage{hyperref}
\usepackage{fancyhdr}
\usepackage{ragged2e}
\usepackage{helvet}
\usepackage{setspace}
\usepackage[paperwidth=8.5in, paperheight=11.0in, top=1.0in, bottom=1.0in, left=1.0in, right=1.0in]{geometry}

% Set Arial (Helvetica) as the default font
\renewcommand{\familydefault}{\sfdefault}
\renewcommand{\sfdefault}{phv}

% Set font size to 12pt
\renewcommand{\normalsize}{\fontsize{12}{14}\selectfont}

% Set line spacing to 1.5
\onehalfspacing

\pagestyle{fancyplain}
\fancyhead[LE,RO]{Activity 1}
\fancyhead[CE,CO]{}
\fancyhead[RE,LO]{P25-LIS3082-2}
\fancyfoot[LE,RO]{\thepage}
\fancyfoot[CE,CO]{Artificial Inteligence, UDLAP}
\fancyfoot[RE,LO]{}

\begin{document}

\fancypagestyle{plain}{
	\renewcommand{\headrulewidth}{1pt}
	\renewcommand{\footrulewidth}{1pt}
}

\renewcommand{\footrulewidth}{1pt}

\title{Class Activity: Gender Bias in Search Engines}
\author{Erick Gonzalez Parada ID: 178145 \\ \vspace{2mm}
Antonio Gutiérrez Blanco ID: 177442 \\ \vspace{2mm}
Andre Francois Duhamel Gutierrez ID: 177315 \\ \vspace{2mm}
Emiliano Ruiz Plancarte ID: 177478 \\ \vspace{2mm}
Jesus Alvarez Sombrerero ID: 177516 \\ \vspace{2mm}
Paulo Stefano Westermann ID: 185573 \\ \vspace{2mm}
Andrés Vizcaya Santacruz ID: 178190
}
\date{\today}


\maketitle

\newpage

\section*{Activity Case 1: Gender Bias in Search Engines}

\subsection*{How do gender biases manifest in search results?}
\begin{justifying}
It considers the contents preferences that the users prefer to watch and consume. So, it depends a lot on the behavior of the users. If the users prefer to sexualize certain genders or look up hate speech the algorithm will prioritize the types of content and favor these topics. There are a lot of examples of gender biased manifestation in search results. For example, in occupational businesses there has been IA tools that prioritized male candidates for job positions, this due to the companies past decisions.
\end{justifying}
\\
\\
\begin{justifying}
Analyzing from another point of view the manifestation of lean over a specific gender in most areas of work is not associated with an immoral ideology, as mentioned before the data will be interpreted such as the inclination is not avoided in any way.
\end{justifying}



\subsection*{What impact do these biases have on society?}
\begin{justifying}
These types of biases generate prejudices that are going to stick around the search engines. Society or the search engine users are going to believe those prejudices and they are going to start growing even more, making the situation even worse.
\end{justifying}
\\
\\
\begin{justifying}
The impact extends to the perception of search engine users, shaping how they view others and themselves. It can create barriers to opportunities, particularly for underrepresented groups, and negatively influence individuals' self-perception, aspirations, and self-esteem.
\end{justifying}
\\
\\
\begin{justifying}
Beyond individual effects, these biases can have broader cultural and economic consequences. Talented individuals may be denied opportunities they deserve, particularly in professional roles, leading to inefficiencies and inequities in the workforce. This normalizes systemic inequalities, hindering societal progress toward fairness and inclusivity.
\end{justifying}



\subsection*{What steps can be taken to minimize these biases in search algorithms?}
\begin{justifying}
	Review training data to ensure adequate representation of women, incorporate data from diverse sources and regions to reduce the overrepresentation of specific demographic groups, and avoid using labels that could reinforce stereotypes.
\end{justifying}


\begin{thebibliography}{9}
	\bibitem{Bains}
	Bains, J. (2018). Research guides: Bias in search engines and algorithms: Home. https://libguides.scu.edu/biasinsearchengines
\end{thebibliography}

\end{document}
