% arara: pdflatex: { shell: yes }
\documentclass[twoside]{article}
\usepackage[utf8]{inputenc}
\usepackage[english]{babel}
\usepackage{amsmath, amssymb, amsthm}
\usepackage{hyperref}
\usepackage{ragged2e}
\usepackage{graphicx}
\usepackage{float}
\usepackage{fancyhdr}
\usepackage{geometry}
\usepackage{multicol}
\usepackage{url}

% Suppress underfull and overfull warnings
\tolerance=1000
\emergencystretch=10pt

\setlength{\headheight}{15.2pt}
\geometry{paperwidth=8.5in, paperheight=11.0in, top=1.0in, bottom=1.0in, left=1.0in, right=1.0in}

\pagestyle{fancyplain}
\fancyhead[LO]{Activity \#6}
\fancyhead[CO]{}
\fancyhead[RO]{P25-LIS3132-2}
\fancyfoot[LO]{\thepage}
\fancyfoot[CO]{Networks and Telecommunications Laboratory, UDLAP}
\fancyfoot[RO]{}

\begin{document}

\fancypagestyle{plain}{
    \renewcommand{\headrulewidth}{1pt}
    \renewcommand{\footrulewidth}{1pt}
}

\title{Activity 6 - Static Routing}
\author{}
\date{\today}
\maketitle

\begin{justify}
    \textbf{\textit{Keywords:}} Static Routing, Router, Network, IP Address, Subnet, Gateway.
\end{justify}

\begin{multicols}{2}
\section{Research-based knowledge}
\subsection*{Goals}
\begin{itemize}
    \item Understand the concept and implementation of static routing
    \item Configure static routes between different network segments
    \item Verify connectivity between hosts in different subnets
    \item Analyze the routing table and trace the path of packets through the network
    \item Compare the advantages and disadvantages of static routing versus dynamic routing
\end{itemize}

\subsection*{Materials}
\begin{itemize}
    \item \textit{Computer with Wi-Fi connection}
    \item \textit{Hyperterminal Software}
    \item \textit{Cisco Packet Tracer or equivalent network simulation software}
    \item \textit{Cisco routers (physical or virtual)}
    \item \textit{Network switches}
\end{itemize}

Static routing is a form of routing where the network administrator manually configures route entries for the router, rather than relying on a dynamic routing protocol to discover routes \cite{geeksforgeeks}. In static routing, routes are explicitly defined by the network administrator and entered into the routing table manually.

Unlike dynamic routing, static routing does not involve the exchange of routing information between routers. Instead, the administrator must update the routing tables manually when the network topology changes \cite{compNetWorks}. Static routes are defined with the following parameters: destination network address, subnet mask, and next-hop address or exit interface.

The main advantages of static routing include improved security (as routing information is not exchanged over the network), reduced CPU and bandwidth usage (as no routing protocol overhead exists), and predictable routing paths (as the administrator has complete control over route selection) \cite{IPCisco}. However, static routing also has disadvantages, such as lack of scalability in large networks, manual reconfiguration requirements when network topology changes, and no automatic adaptation to network failures.

Static routing is commonly used in small networks, for configuring a default route, for connecting to stub networks (networks with only one entry/exit point), and for backup routes when dynamic routing fails \cite{compNetWorks}.


\section{Design of experiments}
The methodology for this laboratory activity consisted of the following steps:

1. \textbf{Network Design}: We designed a network topology with multiple subnets connected by routers. Each subnet was assigned a unique network address with appropriate subnet masks.

2. \textbf{Initial Configuration}: We configured the basic settings on each router, including hostname, interface IP addresses, and subnet masks. We also configured the IP addresses on the end devices in each subnet.

3. \textbf{Static Route Configuration}: On each router, we manually configured static routes to reach remote networks. This involved specifying the destination network address, subnet mask, and next-hop address for each route.

4. \textbf{Default Route Configuration}: For simplicity, we configured default routes on edge routers to forward traffic to unknown destinations.

5. \textbf{Verification}: We verified the configuration by examining the routing tables on each router using the "show ip route" command. We then tested connectivity between devices in different subnets using ping and traceroute commands.

6. \textbf{Documentation}: We documented the network topology, IP addressing scheme, router configurations, and test results for analysis.


\section{Analysis of data}
In this section, we present the results of our static routing implementation and testing. We begin with the network topology design that served as the foundation for our experiments.

\begin{figure}[H]
    \centering
    \includegraphics[width=0.4\linewidth]{imgs/p6-1.jpeg}
    \caption{Network topology design showing the interconnected routers and subnet configuration}
    \label{fig:1}
\end{figure}

As shown in Figure \ref{fig:1}, our network topology consisted of multiple subnets connected through three routers. Each subnet was assigned a unique network address to ensure proper segmentation and routing.

\begin{figure}[H]
    \centering
    \includegraphics[width=0.4\linewidth]{imgs/p6-2.jpeg}
    \caption{Initial router configuration with interface IP addresses and subnet masks}
    \label{fig:2}
\end{figure}

Figure \ref{fig:2} displays the initial configuration of our routers, including the assignment of IP addresses and subnet masks to each interface. This step was crucial for establishing the foundation of our routing infrastructure and ensuring that each router had a unique identifier on the network.

\begin{figure}[H]
    \centering
    \includegraphics[width=0.4\linewidth]{imgs/p6-3.jpeg}
    \caption{Configuration of static routes on Router 1 to reach remote networks}
    \label{fig:3}
\end{figure}

The configuration of static routes on Router 1 is illustrated in Figure \ref{fig:3}. Here, we used the Cisco IOS command syntax to define routes to remote networks that were not directly connected to Router 1. Each static route specified the destination network, subnet mask, and next-hop address to reach that network.

\begin{figure}[H]
    \centering
    \includegraphics[width=0.4\linewidth]{imgs/p6-4.jpeg}
    \caption{Configuration of static routes on Router 2 to reach remote networks}
    \label{fig:4}
\end{figure}

Similarly, Figure \ref{fig:4} shows the configuration of static routes on Router 2. We implemented routes to networks that were not directly connected to Router 2, ensuring that this router could forward packets to all destinations in our network topology.

\begin{figure}[H]
    \centering
    \includegraphics[width=0.4\linewidth]{imgs/p6-5.jpeg}
    \caption{Configuration of static routes on Router 3 to reach remote networks}
    \label{fig:5}
\end{figure}

Figure \ref{fig:5} demonstrates the configuration of static routes on Router 3. As with the other routers, we defined routes to all networks that were not directly connected to Router 3. This completed our static routing configuration across all routers in the topology.

\begin{figure}[H]
    \centering
    \includegraphics[width=0.4\linewidth]{imgs/p6-6.jpeg}
    \caption{Routing table on Router 1 showing configured static routes}
    \label{fig:6}
\end{figure}

After configuring the static routes, we verified our configuration by examining the routing tables on each router. Figure \ref{fig:6} displays the routing table on Router 1, where we can observe both directly connected networks (marked with 'C') and static routes (marked with 'S'). This confirmed that our static route configurations were successfully applied.

\begin{figure}[H]
    \centering
    \includegraphics[width=0.4\linewidth]{imgs/p6-7.jpeg}
    \caption{Routing table on Router 2 showing configured static routes}
    \label{fig:7}
\end{figure}

The routing table on Router 2, shown in Figure \ref{fig:7}, further confirmed our static routing configuration. We observed that all networks in our topology were reachable from Router 2, either through direct connections or via static routes through neighboring routers.

\begin{figure}[H]
    \centering
    \includegraphics[width=0.4\linewidth]{imgs/p6-8.jpeg}
    \caption{Routing table on Router 3 showing configured static routes}
    \label{fig:8}
\end{figure}

Figure \ref{fig:8} presents the routing table on Router 3, completing our verification of the routing tables across all routers. The presence of all network routes in this table confirmed that our static routing configuration was complete and consistent across the entire network.

\begin{figure}[H]
    \centering
    \includegraphics[width=0.4\linewidth]{imgs/p6-9.jpeg}
    \caption{Ping test results showing successful connectivity between hosts in different subnets}
    \label{fig:9}
\end{figure}

To test the functionality of our static routing configuration, we performed ping tests between hosts in different subnets. As shown in Figure \ref{fig:9}, these tests were successful, with all ping packets reaching their destinations and returning to the source. This demonstrated that our static routes were correctly forwarding packets between different network segments.

\begin{figure}[H]
    \centering
    \includegraphics[width=0.4\linewidth]{imgs/p6-10.jpeg}
    \caption{Traceroute results showing the path of packets through the network}
    \label{fig:10}
\end{figure}

Figure \ref{fig:10} displays the results of traceroute commands executed between hosts in different subnets. These results allowed us to visualize the exact path that packets took through our network, confirming that they were following the routes we had defined. The hop counts and transit times provided insights into the efficiency of our routing configuration.

\begin{figure}[H]
    \centering
    \includegraphics[width=0.4\linewidth]{imgs/p6-11.jpeg}
    \caption{Network traffic analysis showing packet flow through the statically routed network}
    \label{fig:11}
\end{figure}

Finally, Figure \ref{fig:11} presents a network traffic analysis that visualizes the flow of packets through our statically routed network. This analysis helped us understand the traffic patterns and verify that packets were being forwarded efficiently according to our static routing configuration. The consistent delivery of packets across all network segments confirmed the success of our implementation.

\section{Synthesis of information}
Through this laboratory activity, we learned that static routing provides a reliable method for establishing communication between different network segments when properly configured. The key findings from our experiments include:

1. Static routing is effective for small to medium-sized networks with a stable topology, as it provides predictable routing paths and reduces protocol overhead.

2. Manual configuration of static routes requires careful planning and precise implementation to avoid routing loops or unreachable networks.

3. The administrative distance of static routes (which is 1 by default) gives them preference over dynamically learned routes, making them useful for creating backup paths or overriding dynamic routing decisions.

4. While static routing is simple to implement in small networks, it becomes increasingly complex and difficult to maintain as the network grows, highlighting the need for dynamic routing protocols in larger networks.

5. Proper documentation of the network topology and routing configuration is essential for troubleshooting and future modifications.

This activity provided valuable hands-on experience with network configuration and reinforced our understanding of IP addressing, subnetting, and routing principles that form the foundation of modern computer networks.

\begin{thebibliography}{9}
    \bibitem{geeksforgeeks}
    Improve, G. (2018, May 4). Types of routing. GeeksforGeeks. from \url{https://www.geeksforgeeks.org/types-of-routing/}
    \bibitem{compNetWorks}
    Static routing configuration guide with examples. (2020, January 12). ComputerNetworkingNotes.
    from \url{https://www.computernetworkingnotes.com/ccna-study-guide/static-routing-configuration-guide-with-examples.html}
    \bibitem{IPCisco}
    What is static routing? (2021, January 4). 
    from \url{IPCisco. https://ipcisco.com/lesson/static-routes/}
\end{thebibliography}

\end{multicols}
\end{document}

